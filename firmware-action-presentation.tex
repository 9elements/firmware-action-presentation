%%%%%%%%%%%%%%%%%%%%%%%%%%%%%%%%%%%%%%%%%
% Focus Beamer Presentation
% LaTeX Template
% Version 1.0 (8/8/18)
%
% This template has been downloaded from:
% http://www.LaTeXTemplates.com
%
% Original author:
% Pasquale Africa (https://github.com/elauksap/focus-beamertheme) with modifications by 
% Vel (vel@LaTeXTemplates.com)
%
% Template license:
% GNU GPL v3.0 License
%
% Important note:
% The bibliography/references need to be compiled with bibtex.
%
%%%%%%%%%%%%%%%%%%%%%%%%%%%%%%%%%%%%%%%%%

%----------------------------------------------------------------------------------------
%	PACKAGES AND OTHER DOCUMENT CONFIGURATIONS
%----------------------------------------------------------------------------------------

\documentclass{beamer}

\usetheme{focus} % Use the Focus theme supplied with the template
% Add option [numbering=none] to disable the footer progress bar
% Add option [numbering=fullbar] to show the footer progress bar as always full with a slide count

% Uncomment to enable the ice-blue theme
%\definecolor{main}{RGB}{92, 138, 168}
%\definecolor{background}{RGB}{240, 247, 255}

% Arch-blue: 23, 147, 209
% Arch-gray: 77, 77, 77
%\definecolor{main}{RGB}{23, 147, 209}
%\definecolor{main}{RGB}{77, 77, 77}
%\definecolor{background}{RGB}{240, 247, 255}

%------------------------------------------------

\usepackage{booktabs} % Required for better table rules
\usepackage{relsize}
\usepackage{xcolor}
\usepackage{listings}
\usepackage{forest}

\definecolor{folderbg}{RGB}{124,166,198}
\definecolor{folderborder}{RGB}{110,144,169}



%----------------------------------------------------------------------------------------
%	 TITLE SLIDE
%----------------------------------------------------------------------------------------

\title{firmware-action}

\subtitle{Build firmware with ease (maybe)}

\author{Vojtech Vesely\texorpdfstring{\\}{,}Marvin Drees}

\titlegraphic{\includegraphics[width=0.4\textwidth,height=0.4\textheight,keepaspectratio]{images/9elements_logo_black.png}}

\date{\today}

\AtBeginSection[]{
	\begin{frame}{Overview}
		\tableofcontents[
			currentsection,
			sectionstyle=show/shaded,
			subsectionstyle=show/show/hide
		]
	\end{frame}
}

\lstdefinestyle{yaml}{
	basicstyle=\color{blue}\tiny,
	rulecolor=\color{black},
	string=[s]{'}{'},
	stringstyle=\color{red},
	comment=[l]{:},
	commentstyle=\color{black},
	%morecomment=[l]{-}
}

\lstdefinestyle{json}{
		string=[s]{"}{"},
		stringstyle=\color{blue},
		comment=[l]{:},
		commentstyle=\color{black},
}


%------------------------------------------------

\begin{document}

%------------------------------------------------

\begin{frame}
	\maketitle % Automatically created using the information in the commands above
\end{frame}


%----------------------------------------------------------------------------------------
%	 About
%----------------------------------------------------------------------------------------

\begin{frame}{About us}
	\begin{figure}
	\centering
	\begin{minipage}{.5\textwidth}
		\Large{Vojtech Vesely}\\
		\small{DevOps}
	\end{minipage}%
	\begin{minipage}{.5\textwidth}
		\Large{Marvin Drees}\\
		\small{Firmware Developer}
	\end{minipage}
	\end{figure}
\end{frame}


%----------------------------------------------------------------------------------------
%	 Table of contents
%----------------------------------------------------------------------------------------

\begin{frame}{Table of contents}
	\tableofcontents[hideallsubsections]
\end{frame}


%----------------------------------------------------------------------------------------
%	 The pain
%----------------------------------------------------------------------------------------

\section{Building firmware is pain (why?)}

\begin{frame}{Building firmware is pain (why?)}
	\begin{figure}
	\centering
	\begin{minipage}{.5\textwidth}
		Problems
		\begin{itemize}
		\item{Multiple firmware options and combinations}
			\begin{itemize}
			\item{\textit{Intel FSP}}
			\item{\textit{coreboot}}
			\item{\textit{edk2}}
			\item{\textit{linux}}
			\item{\textit{uroot}}
			\end{itemize}
		\end{itemize}
	\end{minipage}%
	\begin{minipage}{.5\textwidth}
		\begin{itemize}
		\item{Multiple versions}
			\begin{itemize}
			\item{\textit{coreboot has right now \textbf{32 releases}}}
			\item{\textit{edk2 has right now \textbf{29 releases} and 33 tags}}
			\end{itemize}
		\item{Multiple environments}
			\begin{itemize}
			\item{\textit{environment per version}}
			\item{\textit{environment per OS}}
			\end{itemize}
		\end{itemize}
	\end{minipage}
	\end{figure}
\end{frame}



%----------------------------------------------------------------------------------------
%	 The solution
%----------------------------------------------------------------------------------------

\section{firmware-action (what?)}

\begin{frame}{firmware-action (what?)}
	What it is?
	\begin{itemize}
	\item{\textbf{Makefile} or \textbf{Taskfile} but for firmware}
	\end{itemize}
	\vspace{20px}

	Key features:
	\begin{itemize}
	\item{unified environment}
	\item{build in CI/CD = build in local}
	\item{written on Golang}
	\item{GitHub action integration}
	\item{AUR package for Arch Linux users}
	\end{itemize}

	\centering
	\includegraphics[width=0.5\textwidth,height=0.5\textheight,keepaspectratio]{images/golang-logo.png}
\end{frame}

\begin{frame}{firmware-action (what?)}
	\begin{figure}
	\centering
	\begin{minipage}{.5\textwidth}
		Before:
		\vspace{20px}

		\begin{tabular}{ |c|c| }
			\hline
			\textbf{CI} & \textbf{local} \\
			\hline
			\hline
			build.yaml & build.sh \\
			\hline
			\multicolumn{2}{|c|}{code + config} \\
			\hline
		\end{tabular}
	\end{minipage}%
	\begin{minipage}{.5\textwidth}
		After:
		\vspace{20px}

		\begin{tabular}{ |c| }
			\hline
			\textbf{CI \& local} \\
			\hline
			\hline
			firmware-action + config \\
			\hline
			code + config \\
			\hline
		\end{tabular}
	\end{minipage}
	\end{figure}
\end{frame}



%----------------------------------------------------------------------------------------
%	 dagger our savior
%----------------------------------------------------------------------------------------

\section{Docker + dagger (how?)}

\subsection{What is Docker}
\begin{frame}{What is Docker}
	\begin{itemize}
	\item{lightweight container}
	\item{For example}
		\begin{itemize}
		\item{Ubuntu 20.04 ~~~~ \textit{(current: 24.10)}}
		\item{nginx 1.18.0 ~~~~~~ \textit{(current: 1.27.2)}}
		\item{PostgreSQL 12.20 ~ \textit{(current: 17.2)}}
		\end{itemize}
	\end{itemize}
\end{frame}

\subsection{What is Dagger}
\begin{frame}{What is Dagger}
	\begin{figure}
	\centering
	\begin{minipage}{.5\textwidth}
		\begin{itemize}
		\item{docker orchestrator}
		\item{programmable CI/CD engine}
		\item{\url{docs.dagger.io}}
		\end{itemize}
	\end{minipage}%
	\begin{minipage}{.5\textwidth}
		For example:
		\begin{itemize}
		\item{pull \textit{this} docker container}
		\item{copy \textit{this} directory into container}
		\item{run \textit{this} command in container}
		\end{itemize}
	\end{minipage}
	\end{figure}
\end{frame}



%----------------------------------------------------------------------------------------
%	 Example
%----------------------------------------------------------------------------------------


\section{Example}

\begin{frame}{Example: JSON config}
\lstinputlisting[style=json, basicstyle=\tiny]{firmware-action-example/coreboot-example.json}
\end{frame}

\begin{frame}{Example: GitHub YAML}
\lstinputlisting[style=yaml]{firmware-action-example/.github/workflows/coreboot-example.yml}
\end{frame}

\begin{frame}{Example: Local execution}
\lstinputlisting[basicstyle=\tiny]{code\_examples/workflow\_new.cmd}
\end{frame}

\begin{frame}{Example: Local execution: Taskfile}
\lstinputlisting[style=yaml]{firmware-action-example/Taskfile.yml}
\end{frame}


%----------------------------------------------------------------------------------------
%	 Demo
%----------------------------------------------------------------------------------------

\section{Demo}
\begin{frame}{Demo}
	\begin{enumerate}
	\item{Get firmware-action}
		\begin{itemize}
		\item{AUR}
		\item{clone \url{https://github.com/9elements/firmware-action.git}}
		\end{itemize}
	\item{Clone firmware-action-example}
		\begin{itemize}
		\item{clone \url{https://github.com/9elements/firmware-action-example.git}}
		\end{itemize}
	\end{enumerate}
\end{frame}


%----------------------------------------------------------------------------------------
%	 Links and sources
%----------------------------------------------------------------------------------------

\section{Sources}
\begin{frame}{Sources}
	\begin{figure}
	\centering
	\includegraphics[width=0.7\textwidth,height=0.5\textheight,keepaspectratio]{images/firmware-action-qr.png}
	\caption{Code: \url{github.com/9elements/firmware-action}}
	\label{fig:link-firmware-action}
	\end{figure}
\end{frame}


%----------------------------------------------------------------------------------------

\end{document}
